% ---------------------------------------------------------
% Project: PhD KAPPA
% File: abstract.tex
% Author: Andrea Discacciati
%
% Purpose: Abstract frontmatter
% ---------------------------------------------------------

\noindent {\Large \textsf{\textbf{{Abstract}}}
\bigskip
%\chapter*{Abstract}
\small
\par Prostate cancer is the second most common cancer among men worldwide, yet its etiology remains poorly understood. Obesity, on the other hand, is a prevalent but preventable medical condition that is associated with hormonal and metabolic changes. Since prostate cancer is a hormone-related cancer, the hypothesis of a link between body fatness and prostate cancer risk has been formulated. Furthermore, the considerable biologic heterogeneity of prostate cancer warrants analyses to be carried out separately by aggressiveness of the disease, differentiating indolent from  potentially lethal tumors.

This thesis has two aims. First, to elucidate the association between obesity, as measured by body mass index (BMI), and the risk of localized, advanced, and fatal prostate cancer. This is done using both primary data \citepalias{discacciati_body_2011} and aggregated data extracted from published epidemiological studies \citepalias{discacciati_body_2012}. Second, to deal with some methodological aspects related to the analysis of primary and aggregated data \citepalias{bellavia_using_2015, discacciati_interpretation_2015, discacciati_goodness_2015}. 

In \citetalias{discacciati_body_2011}, we used primary data from the Cohort of Swedish Men to examine the association of BMI during early adulthood (30 years of age) and middle-late adulthood (45–79 years of age) with the incidence of localized and advanced prostate cancer and with prostate cancer mortality. %During 11 years of follow-up for incident and 10 years for fatal prostate cancer, we observed 1,XXX localized, XXX advanced, and XXX fatal cases. 
BMI during middle-late adulthood was observed to be inversely associated with the incidence of localized prostate cancer, while it was directly associated with the incidence of advanced prostate cancer and with prostate cancer mortality. At the same time, we observed limited evidence of an inverse association between BMI during early-adulthood and the risk of advanced and fatal prostate cancer.

In \citetalias{bellavia_using_2015}, we extended the use of quantile regression for censored data to those situations where the time scale of interest is attained age at the event instead of follow-up time.  In particular, we described how to use Laplace regression to model percentiles of age at the event in the presence of delayed entries, by conditioning on age at entry.  %We also discussed the consequences that changing the time scale has on the interpretation of survival curve and survival percentiles.

In \citetalias{discacciati_interpretation_2015}, we identified three major misinterpretations of risk and rate advancement periods (RAP): first, equating RAP with the difference in mean survival times; second, interpreting RAP as the time by which the survival curve for the exposed individuals is shifted compared with that for the unexposed; third, equating the RAP to a simple ratio of two log--relative risks. Furthermore, we showed how RAP estimation is sensitive to the specification of the age-disease association.

In \citetalias{discacciati_body_2012}, we carried out a dose--response meta-analysis to summarize the available evidence on the association between BMI during middle-late adulthood and the incidence of localized and advanced prostate cancer. Based on aggregated data extracted from 13 prospective studies, we observed that BMI was inversely associated with the incidence of localized prostate cancer, while it was directly associated with the incidence of advanced prostate cancer.

In \citetalias{discacciati_goodness_2015}, we stressed the importance of assessing the goodness of fit of dose--response meta-analysis models. We presented and discussed three tools (deviance, coefficient of determination, and \rveplot) that are useful to test, quantify, and visually display the fit of dose--response meta-analysis models, while taking into account the correlation structure of the study-specific log--relative risks.

In conclusion, \citetalias{discacciati_body_2011} and \citetalias{discacciati_body_2012} supported the hypothesis of etiological heterogeneity of prostate cancer in relation to obesity during middle-late adulthood. In particular, BMI was observed to be directly associated with advanced prostate cancer and with prostate cancer mortality. \citetalias{bellavia_using_2015} extended the use of quantile regression for censored data to those situations where attained age is the time scale of interest, \citetalias{discacciati_interpretation_2015} clarified the appropriate use and interpretation of RAP, and \citetalias{discacciati_goodness_2015} proposed useful and relevant methods for assessing the goodness of fit of dose–response models in research synthesis.
\normalsize


