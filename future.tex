% ---------------------------------------------------------
% Project: PhD KAPPA
% File: future.tex
% Author: Andrea Discacciati
%
% Purpose: Future research
% ---------------------------------------------------------

\chapter{Future research}

Based on the conclusions presented in this thesis, future research includes: 

\begin{itemize}
\item Further investigating the role on prostate cancer incidence and mortality of obesity measured at different time points in life. Cohort studies of children, adolescents, or young adults initiated decades ago could for example provide invaluable data to elucidate these associations. Furthermore, linkage of cross-sectional or cohort data collected decades ago with cancer registries could be another viable short-term option. Records of children's weight collected at school or military enlistment registries (military service was mandatory in Sweden between 1901 and 2010) could provide unique data on anthropometric measurements.

\item More in general, body fatness and changes in body weight over the life course in relation to prostate cancer risk deserve additional investigation. This requires well-designed prospective studies of prostate cancer--free men followed to diagnosis and to death. Availability of repeated measurements is essential to be able to answer this question. Moreover, differences between weight gain/loss around the time of diagnosis and long-term weight gain/loss should be considered when focusing on prostate cancer mortality. Lastly, one should be able to distinguish between intentional and unintentional weight loss. Data from the COSM, thanks to the self-reported weight measurements at two time points (1997 and 2008),  could help answering some of these research questions.

\item Evaluating the role played by detection bias and other alternative explanations to biological mechanisms in the association between obesity and incidence of localized and advanced prostate cancer. 

\item Extending the use of survival percentiles to measures that assess the public health impact of an exposure. For example, one could think of a percentile-based measure analogous to the excess fraction \citep{greenland_measures_2008}, where a given PD is recalculated as a fraction of the survival percentile among the exposed. % could be defined as 
%\begin{equation*}
%\frac{Q_T(p|e=0)-Q_T(p|e=1)}{Q_T(p|e=0)}
%\end{equation*}
%and interpreted as the fraction of the $100p$-th survival percentile attributable to the exposure among the exposed. %TODO: check last sentence

\item Regression methods for measures of disease occurrence based on survival percentiles, such as the geometric rate regression model \citep{bottai_regression_2015}, merit further attention both from an applied and methodological point of view.

\item Extending and applying additive models \citep{buja_linear_1989} to dose--response meta-analysis. This would provide  a flexible tool to investigate the shape of the dose-risk relation. Formulae for fitting additive models by penalized least squares need to be modified to take into account the correlation structure of the error term and  the lack of the intercept. Additive models can be used together with the one-stage meta-analysis approach \citepalias[see][]{discacciati_goodness_2015}.

\item Developing a point-wise average approach for dose--response meta-analysis, extending the work carried out by \citet{sauerbrei_new_2011} on meta-analysis of individual patient data. One of the possible advantages that this approach would give is that the exposure transformation is not constrained to be the same across all the individual studies, as it is the case in the two-stage approach presented in this thesis. 

\item As a next step following the goodness of fit tools introduced in \citetalias{discacciati_goodness_2015}, it would be valuable to develop outlier and influence diagnostics for dose--response meta-analysis. These diagnostic tools would prove useful in sensitivity analyses to assess robustness and stability of dose--response meta-analysis models. 

\end{itemize}