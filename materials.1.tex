% ---------------------------------------------------------
% Project: PhD KAPPA
% File: materials.1.tex
% Author: Andrea Discacciati
%
% Purpose: paper 1 (materials)
% ---------------------------------------------------------


\section{Paper I}

\subsubsection{Exposure assessment}

BMI during early adulthood was calculated as recalled weight (in kilograms) at the age of 30 years multiplied by self-reported height (in meters) to the power of minus 2. In a similar fashion, BMI during middle-late adulthood was computed from the self-reported weight at baseline --- that is, at ages 45--79 years --- and self-reported height. In addition to the exclusions described in section \ref{section:cosm}, we excluded those men with  BMI at baseline or at 30 years of age outside the range 15--40 \kgmsq{} ($n=196$) or missing ($n=8751$), leaving thus 36,959 men for the analyses. 

\subsubsection{Outcome assessment}

Incident cases of prostate cancer were ascertained by linkage with the SCR, while linkage with the NPCR provided data regarding TNM staging, Gleason score, and PSA serum level at diagnosis. Prostate cancer cases were classified by subtype of the disease either as localized [T1--2 \textit{and} Nx--0 \textit{and} (Mx--0 \textit{or} PSA < 20 \ngml{} \textit{or} Gleason score 2--7)] or advanced [T3--4 \textit{and} Nx--1 \textit{and} (Mx--1 \textit{or} PSA > 100 \ngml{} \textit{or} Gleason score 8--10)]. These classification criteria do not exactly match with those used by the \citet{npcr_prostatacancer_2013}. Nevertheless, the definition of localized cases is comparable to the combination of `low-risk' and `intermediate-risk' categories, whereas the definition of advanced cases is comparable to the combination of `regionally metastatic' and `distant metastases' categories (table \ref{table:classifpca}). Information on prostate cancer deaths was obtained from the CDR.

From 1 January 1998 to 31 December 2008, during 371,792 person-years, a total of 1,530 localized and 554 advanced cases were identified. From 1 January 1998 to 31 December 2007, during 333,702 person-years, 225 cases of prostate cancer death were documented.

\subsubsection{Statistical analysis}

Cox PH models were used to examine the multivariable-adjusted association of both BMI at 30 years of age and BMI at baseline age with the IR or MR of prostate cancer, as appropriate. The exposures were modeled as continuous variables and second-degree Fractional Polynomials (FP2) \citep{royston_use_1999} were employed whenever this provided a better overall fit of the model, as measured by the Akaike Information Criterion (AIC). The value of 22 \kgmsq{} was used as the referent. Models were adjusted for age (years), total energy intake (kcal), total physical activity (<37.9, 38.0--40.9, 41.0--44.9, $\ge$45.0 MET-h/day), education (years), smoking status (current, former, never smoker), family history of prostate cancer (yes, no, don't know), and personal history of diabetes (yes, no).

\subsubsection{Updated analysis}

In section~\ref{section:updated_paper1}, we present updated analyses using the most recent data available at our Unit to take advantage of the extended follow-up period.  

The number of subjects in the study ($n=36959$) and the classification criteria for localized and advanced subtypes of prostate cancer were left unchanged. The follow-up was extended until 31 December 2011 for incidence of prostate cancer subtypes, and until 31 December 2012 for prostate cancer mortality. During 456,322 person-years, a total of 2,078 localized cases and 727 advanced cases were identified. On the other hand, during 500,765 person-years, 508 deaths attributed to prostate cancer were documented. 

Multivariable-adjusted Cox PH models, similar to those previously described, were employed. In these updated analyses, however, the baseline hazard was allowed to vary between counties of residence at baseline (Västmanland/Örebro) by means of stratified Cox models \citep{vanhouwelingen_cox_2013}. Results were presented in tabular form by categorizing BMI in 6 groups: < 21.0, 21.0--22.9 (referent), 23.0--24.9, 25.0--27.4, 27.5--29.9, and $\ge 30.0$ \kgmsq{}. 

In the analysis on localized prostate cancer, BMI at baseline age was again modeled using FP2. Moreover, RCS with 4 knots placed at the 5th, 35th, 65th, and 95th percentiles of the distributions of BMI at baseline age and of BMI at age 30 years were also employed for all the analyses.\footnote{Corresponding to knots placed at 20.9, 24.3, 26.5, and 31.6 \kgmsq{} for BMI at baseline age, and at 19.4, 22.1, 23.8, and 27.4 \kgmsq{} for BMI at age 30 years.} Lastly, we examined whether the associations differed according to county of residence at baseline by including appropriate product terms in the Cox models. 
