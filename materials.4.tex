% ---------------------------------------------------------
% Project: PhD KAPPA
% File: materials.4.tex
% Author: Andrea Discacciati
%
% Purpose: Section meta-analysis paper 4 (materials)
% ---------------------------------------------------------

\section{Paper IV}

\subsubsection{Literature search and study characteristics}

Identification of the studies reporting information about the association between BMI and incidence of localized and advanced prostate cancer was done by searching the Medline (PubMed) and Embase databases. The search query was the following: (obesity \textit{or} BMI \textit{or} ``body size'' \textit{or} adiposity) \textit{and} ``prostate cancer''. Moreover, reference lists from reviews and other relevant publications were reviewed to identify further studies to be included in the meta-analysis. Lastly, effort was put in the identification of possible studies not included in the computerized databases. % or, more in general, of unpublished results. Searches were limited to articles written in English language.

A total of 15 articles were left after the exclusion of case-control studies ($n=54$), studies not reporting results separately by prostate cancer subtypes ($n=16$), and duplicate studies on the same population ($n=2$). In addition, 2 out of these 15 studies were excluded because did not report the number of cases and person-years per BMI category \citep{habel_body_2000, gong_obesity_2006}, which is necessary to approximate the study-specific variance-covariance matrix $\mathbf{S}_i$. 

Consequently, 13 prospective studies were available for the analyses \citep{cerhan_association_1997, giovannucci_height_1997, putnam_lifestyle_2000, schuurman_anthropometry_2000, macinnis_body_2003, kurahashi_association_2006, littman_anthropometrics_2007, rodriguez_body_2007, wright_prospective_2007, pischon_body_2008, wallstrom_prospective_2009, stocks_blood_2010, discacciati_body_2011}, 1 of which reported results only for advanced prostate cancer \citep{giovannucci_height_1997}, leaving thus 12 articles for the meta-analysis on localized prostate cancer. A flowchart summarizing the identification of relevant studies is shown in  \citetalias[figure 1]{discacciati_body_2012}, while detailed information about the single studies is available in \citetalias[supplemental table~S1]{discacciati_body_2012}.

The 12 studies on localized prostate cancer, published during the period 1997--2011, involved a total of 1,033,009 men and 19,130 incident cases. Five of these studies were conducted in the U.S., 3 in Sweden, 1 in the Netherlands, 1 in Australia, 1 in Japan, and 1 was an European multi-centre study. Four studies relied on trained personnel to collect weight and height, while the remaining 8 studies were based on self-reported measurements. Lastly, three studies provided RR estimates adjusted for both physical activity and personal history of diabetes, while 9 studies adjusted only for one of these two variables or for neither. 

The 13 studies on advanced prostate cancer included a total of 1,080,790 men and 7,067 newly diagnosed cases. The study that reported results only for advanced prostate cancer was conducted in the U.S, relied on self-reported weigh and height measurements, and did not adjust for physical activity activity or for personal history of diabetes.

Interestingly, the criteria used to classify incident prostate cancer cases as localized or advanced were very heterogeneous. In particular, they were based on the Gleason score, World Health Organization grading system, TNM and Jewett--Whitmore staging system, PSA level, or different combinations of these. 

\subsubsection{Updated dose--response meta-analysis}

For localized prostate cancer, the studies by \citet{macinnis_body_2003} [Melbourne Collaborative Cohort Study (MCCS)] and \citetalias{discacciati_body_2011} (COSM) were superseded by \citet{bassett_weight_2012} and by the updated analyses presented in section \ref{section:updated_paper1}, respectively. The study by \citet{gong_obesity_2006} was included assuming 0 correlation between the logRRs. The study by \citet{schuurman_anthropometry_2000} was excluded because it reported only a linear trend. Lastly, 2 studies not included in \citetalias{discacciati_body_2012} were added to the meta-analysis \citep{hernandez_relationship_2009, grotta_physical_2015}. Thus, a total of 14 prospective cohort studies were available for the analysis, including a total of 1,081,926 and 26,500 cases.

For advanced prostate cancer, the studies by \citet{macinnis_body_2003} (MCCS) and \citetalias{discacciati_body_2011} (COSM) were superseded by \citet{bassett_weight_2012} and by the analyses presented in section \ref{section:updated_paper1}, respectively. The studies by \citet{habel_body_2000} and by \citet{gong_obesity_2006} were included assuming that the reported logRR estimates were uncorrelated. The study by \citet{schuurman_anthropometry_2000} was excluded because it reported only a linear trend. Lastly, 4 studies were added to the meta-analysis \citep{hernandez_relationship_2009, shafique_cholesterol_2012, grotta_physical_2015, moller_prostate_2015}. In summary, the meta-analysis on advanced prostate cancer was based on 18 prospective studies including 1,240,222 men and 10,174 incident cases.
